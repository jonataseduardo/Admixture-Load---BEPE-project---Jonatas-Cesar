\documentclass[a4paper, 12pt]{article}

%\usepackage[portuges]{babel}
\usepackage[utf8]{inputenc}
\usepackage{amsmath,amssymb}
\usepackage{indentfirst}
\usepackage{libertine}
\usepackage{fancyhdr}
\usepackage{fullpage}
\usepackage{setspace}
\usepackage{natbib}
\usepackage[hidelinks]{hyperref}
\usepackage{url}

\renewcommand{\baselinestretch}{1.5} 

\begin{document}

\section*{Justification for the choice of the research group}

The research group headed by the Prof. John Novembre at the Department of
Human Genetics at the University of Chicago has the expertise in the use of
computational approaches to investigate the evolutionary processes that have
shaped the human genetic diversity
[{\footnotesize \url{https://genes.uchicago.edu/directory/john-novembre-phd}}]. 

In the past ten years the Prof. John Novembre have published over 75 papers and
accumulated over 8400 citations [{\footnotesize \url{http://jnpopgen.org/}}].
Some of the most cited sudies of Prof. John November includes papers such as
\textit{"Fast model-based estimation of ancestry in unrelated individuals"}
\citep{Alexander2009}, which introduces the software ADMIXTURE and the paper
\textit{"Interpreting principal component analyses of spatial population
  genetic variation"} \citep{Novembre2008}. 

Some of the recent works of the Prof John Novembre that are related with
admixture processes and genetic load are \textit{"Visualizing spatial
  population structure with estimated effective migration surfaces"}
\citep{Petkova2016} and respectively \textit{"A Temporal Perspective on the
  Interplay of Demography and Selection on Deleterious Variation in Humans"}
\citep{Koch2017a}.

\bibliographystyle{abbrvnat85}
\bibliography{mend}

\end{document}
